%% define the main template style for the document
\documentclass{article}
%% import the NIPS package for LaTeX style
\usepackage[final]{nips_2017}

%% import packages that have custom options
\usepackage[utf8]{inputenc}
\usepackage[T1]{fontenc}
\usepackage[pagebackref=true]{hyperref}
\usepackage[nolist,nohyperlinks]{acronym}

%% Import general packages
\usepackage{
  amsmath, amssymb, amsfonts, nicefrac,
  algorithmic, textcomp, listings, url,
  graphicx, subfig, microtype,
  booktabs, longtable
}

%% start the document, the Markdown parser takes over from this point
\begin{document}
\title{Review: A Neural Algorithm of Artistic Style}

%% TODO: notations section?
%% TODO: clarify notations for both content and styl representation (N_l etc.)

\author{
    James C. Kauten \\
    Department of Software Engineering \\
    Auburn University \\
    Auburn, AL 36832 \\
    \texttt{kautenja@auburn.edu} \\
    \And
    Behnam Rasoolian \\
    Department of Industrial Engineering \\
    Auburn University \\
    Auburn, AL 36832 \\
    \texttt{bzr0014@auburn.edu} \\
}

\maketitle

\section{Paper Summary}

In their arXiv preprint, \textit{A Neural Algorithm of Artistic Style},
\cite{2015arXiv150806576G} prove a level of separability between the
\textit{content} of an image and the \textit{style} that characterizes it.
Using a \ac{CNN} trained to classify images on the ImageNet benchmark, they
transfer the style of famous works of art onto the content of arbitrary
photographs. They define loss functions between the activation maps of
various layers in the network that measure either content or style loss.
Minimizing the joint loss between a noise image $\textbf{x}$ and both a
content image $\textbf{p}$ and style image $\textbf{a}$ transfers the global
features of $\textbf{p}$, with the local styles of $\textbf{a}$ onto the
noise image $\textbf{x}$. Put simply, their algorithm paints photographs using
arbitrary works of art as a palette for colors and textures.


\subsection{Content}

\subsubsection{Representation}

This work relies heavily on the understanding of convolutional layers. As a
collection of image filters, each layer extracts unique features from its
input image. As such, \cite{2015arXiv150806576G} postulate that as the layer
depth increases, the network cares more about the content of the image. That
is to say, deeper layers have a more specific understanding of what composes
a given image. Whereas, the shallower layers primarily understand the image
as raw pixels. This leads to their definition of \textit{content
representation} as the activations from deep layers in the network.

To objectively measure the difference of content between two images,
\cite{2015arXiv150806576G} define a loss function $\mathcal{L}_{content}$.
Given a content image $\textbf{p}$, a noise image $\textbf{x}$, and an
arbitrary layer $l$, the activations at $l$ for $\textbf{p}$ and $\textbf{x}$
are defined as $P^l$ and $F^l$ respectively. The squared euclidean distance
then measures the $\mathcal{L}_{content}$ loss between $P^l$ and $F^l$. Eq.
\ref{eq:content-loss} shows this loss function with an additional factor of
$\frac{1}{2}$ to simplify the formulation of the analytical gradient in Eq.
\ref{eq:content-grad}.

% TODO: note the M_l and N_l variables in the above paragraph
\begin{equation}
\label{eq:content-loss}
\mathcal{L}_{content}(\mathbf{p}, \mathbf{x}, l) =
\frac{1}{2} \sum_{i=1}^{N_l}\sum_{j=1}^{M_l}{(F^l_{ij} - P^l_{ij})^2}
\end{equation}

\begin{equation}
\label{eq:content-grad}
\frac{\partial \mathcal{L}_{content}}{\partial F^l_{ij}} =
\begin{cases}
    (F^l - P^l)_{ij} & \iff F^l_{ij} > 0 \\
    0 & \iff F^l_{ij} < 0 \\
\end{cases}
\end{equation}

\subsubsection{Reconstruction}

By back propagating the error from the $\mathcal{L}_{content}$ loss, we can
minimize the difference between a content image $\textbf{p}$ and a noise image
$\textbf{x}$ based on a given layer $l$. \cite{2015arXiv150806576G} find that
the second convolutional layer of the fourth block (\textit{block4\_conv2}) of
VGG19 produces the most desirable results.
% Fuck this sentence
Naturally, this is assessment is subjective to the beholders tastes.

To better understand the raw effect of this loss metric, we reconstructed an
image by taking the content loss at five separate layers in VGG19. We use the
first convolutional layer from each of the five blocks to produce the five
content reconstructions depicted in Fig. \ref{fig:content-reconstruction}. We
can see that the first three layers (block1\_conv1, block2\_conv1,
block3\_conv1) reproduce the input image in nearly full detail. However, the
fourth and fifth layers (block4\_conv1, block5\_conv1) reconstruct the image
with looser form. Deeper layers tend to work better for style transfer because
this looser representation allows the content to blend more smoothly with
other images while still preserving the global features of the content.

\begin{figure}[htp]
\centering
\caption{Content Reconstruction of \textit{T\"{u}bingen, Germany}}
\label{fig:content-reconstruction}

    \begin{minipage}{0.3\linewidth}
    \includegraphics[width=\textwidth]{img/content/block1_conv1}
    \captionof*{figure}{block1 conv1}
    \end{minipage}
    \begin{minipage}{0.3\linewidth}
    \includegraphics[width=\textwidth]{img/content/block2_conv1}
    \captionof*{figure}{block2 conv1}
    \end{minipage}
    \begin{minipage}{0.3\linewidth}
    \includegraphics[width=\textwidth]{img/content/block3_conv1}
    \captionof*{figure}{block3 conv1}
    \end{minipage}

\medskip

    \begin{minipage}{0.3\linewidth}
    \includegraphics[width=\textwidth]{img/content/block4_conv1}
    \captionof*{figure}{block4 conv1}
    \end{minipage}
    \begin{minipage}{0.3\linewidth}
    \includegraphics[width=\textwidth]{img/content/block5_conv1}
    \captionof*{figure}{block5 conv1}
    \end{minipage}

\end{figure}



\subsection{Style}

\subsubsection{Representation}

Much like the content representation, style representation relies on the
feature responses of particular layers in the \ac{CNN}. However, this
representation uses a different feature space. Converting each activation
map to a \textit{gram matrix} allows the extraction of just the texture from
a given image. It does so by computing the correlations between different
filters in an arbitrary convolutional layer $l$. Simply put, the gram matrix
$G^l$ for an activation map is the inner product of feature maps:

\begin{equation}
G_{i j}^l = \sum_{k}^{M_l} F_{i k}^l F_{j k}^l
\end{equation}

With a new feature space representation of raw texture,
\cite{2015arXiv150806576G} define an additional loss function
$\mathcal{L}_{style}$ between an artwork image $\textbf{a}$, and a noise image
$\textbf{x}$ for some convolutional layer $l$. First, the activations at $l$
for $\textbf{a}$ and $\textbf{x}$ are transformed to their respective gram
matrices $A^l$, and $G^l$. Then, much like the content loss, we define
the style loss for a given layer as the squared euclidean distance between the
gram matrices $A^l$, and $G^l$:

\begin{equation}
E_l =
\frac{1}{4 N_l^2 M_l^2}
\sum_{i=1}^{N_l}\sum_{j=1}^{M_l}
(G^l_{ij} - A^l_{ij})^2
\end{equation}

\cite{2015arXiv150806576G} incorporate multiple layers in the style loss using
a weighted sum. In their experiments, they simply use a static weight
$w_l = \frac{1}{L}$ where $L$ is the number of layers contributing to the
style loss. Eq. \ref{eq:style-loss} displays the final formulation of the
$\mathcal{L}_{style}$ loss between $\textbf{a}$ and $\textbf{x}$ for some set
of layers bounded by $L$.

\begin{equation}
\label{eq:style-loss}
\mathcal{L}_{style}(\mathbf{a}, \mathbf{x}, L) = \sum_{l=0}^L w_l E_l
\end{equation}

Finally, we derive the gradient of this loss metric in Eq.
\ref{eq:style-grad}. Again, this gradient back propagates through the network
to provide a gradient of the loss with respect to the noise image
$\textbf{x}$.

\begin{equation}
\label{eq:style-grad}
\frac{\partial E_l}{\partial F^l_{ij}} =
\begin{cases}
    \frac{1}{N^2_l M^2_l}((F^l)^T (G^l - A^l))_{ji} & \iff F^l_{ij} > 0 \\
    0 & \iff F^l_{ij} < 0 \\
\end{cases}
\end{equation}

\subsubsection{Reconstruction}

\cite{2015arXiv150806576G} use the first convolutional layer of each block
(yielding five layers total) in their style loss. The selection of layers
for style reconstruction is similar to content reconstruction in that it is
subjective. To visualize the implications of layer selection, we reconstruct
style using five different sets of layers from VGG19. Fig.
\ref{fig:style-reconstruction} shows the output based on each of the five
sets. Unlike the content reconstruction, the style reconstruction preserves
none of the global content. Instead, it selects raw colors and textures from
the style, rearranging them around the new canvas $\textbf{x}$.

Like the content loss, additional layers provide a more in-depth understanding
of the style. The first convolutional layer of the first block seems to
extract simple points of color based on the color distribution of the style
$\textbf{a}$. As more layers contribute to the loss, the details of the
texture spread and smoothen across the noise image $\textbf{x}$.

\begin{figure}[htp]
\centering
\caption{Style Reconstruction of Vincent Van Gogh's \textit{A Starry Night}}
\label{fig:style-reconstruction}

    \begin{minipage}{0.3\linewidth}
    \includegraphics[width=\textwidth]{img/style/block1_conv1}
    \captionof*{figure}{block1 conv1}
    \end{minipage}
    \begin{minipage}{0.3\linewidth}
    \includegraphics[width=\textwidth]{img/style/block2_conv1}
    \captionof*{figure}{block1,2 conv1}
    \end{minipage}
    \begin{minipage}{0.3\linewidth}
    \includegraphics[width=\textwidth]{img/style/block3_conv1}
    \captionof*{figure}{block1,2,3 conv1}
    \end{minipage}

\medskip

    \begin{minipage}{0.3\linewidth}
    \includegraphics[width=\textwidth]{img/style/block4_conv1}
    \captionof*{figure}{block1,2,3,4 conv1}
    \end{minipage}
    \begin{minipage}{0.3\linewidth}
    \includegraphics[width=\textwidth]{img/style/block5_conv1}
    \captionof*{figure}{block1,2,3,4,5 conv1}
    \end{minipage}

\end{figure}



\subsection{Transfer}




\section{Strengths}

As the preliminary paper on neural style transfer, the strength of this work
lies in the novelty of the problem \& solution. \cite{2015arXiv150806576G}
loosely define a new problem of \textit{style transfer} and present an
optimization method for solving it.
% this sentance sucks. pull it out and replace it
Additionally, the problem raises new questions as to what
composes content vs style and where to draw the line between the two.

Despite involving complex technology, the actual method is beautifully simple.
Utilizing a network pre-trained to detect objects provides a strong analog to
a human artist. Different layer, hyperparameter, and initial noise selections,
resembles how different humans may experience similar things, but produce
vastly different artistic styles. Their work may not focus on neuro-science,
but it provides a unique window into a theory of how visual creativity might
function on a base level.



\section{Weaknesses}

Although the work of \cite{2015arXiv150806576G} shows many strengths, it
demonstrates a level of weakness as well. Because the problem they're solving
is loosely defined and largely based on subjective opinion, a truly objective
measure of either methods or solutions is unattainable. As such, comparing new
methods to this one presents a challenge for future work to dismantle.

Despite reporting the best results using L-BFGS to optimize noise,
\cite{2015arXiv150806576G} present no substantiation or comparative metrics
against other optimization techniques. It's understandable that they might
gloss over this detail given the subjectivity of the end result. But, it
seems to lack scientific rigor. We find that the Adam optimizer produces
better results at a fraction of the complexity, both computational, and
mathematical.



\section{Future Work}




\section{Questions \& Answers}

\paragraph{Layers Used in Loss Functions} \textit{Why do [the authors] use 5
convolutional layers? Could there be more or less?}

\cite{2015arXiv150806576G} select the 5 style layers based on their results
for style reconstruction. The five layers in question provide results that
they find the most aesthetically pleasing. That said, there can be as few as
1 layer used for style loss or as many as necessary. Fig.
\ref{fig:style-layers-effect} shows the effect of different sets of style
layers in the style loss. We see that the additional layers do in fact help
transfer more of the style.

\begin{figure}[htp]
\centering
\caption{Samford Hall Styled as Pablo Picasso's \textit{Seated Nude} Using
Different Style Loss Layer Sets}
\label{fig:style-layers-effect}

    \begin{minipage}{0.3\linewidth}
    \includegraphics[width=\textwidth]{img/style-layer-selection/block1_conv1}
    \captionof*{figure}{block1 conv1}
    \end{minipage}
    \begin{minipage}{0.3\linewidth}
    \includegraphics[width=\textwidth]{img/style-layer-selection/block2_conv1}
    \captionof*{figure}{block1,2 conv1}
    \end{minipage}
    \begin{minipage}{0.3\linewidth}
    \includegraphics[width=\textwidth]{img/style-layer-selection/block3_conv1}
    \captionof*{figure}{block1,2,3 conv1}
    \end{minipage}

\medskip

    \begin{minipage}{0.3\linewidth}
    \includegraphics[width=\textwidth]{img/style-layer-selection/block4_conv1}
    \captionof*{figure}{block1,2,3,4 conv1}
    \end{minipage}
    \begin{minipage}{0.3\linewidth}
    \includegraphics[width=\textwidth]{img/style-layer-selection/block5_conv1}
    \captionof*{figure}{block1,2,3,4,5 conv1}
    \end{minipage}

\end{figure}

For the content loss, they selection block4 conv2 because they find the
representation at this activation map allows the content to blend better with
the style. This provides a better "painting" effect of the content by
smoothing out sharp lines and corners as if they were placed by a paint brush.
Fig. \ref{fig:content-layers-effect} shows the effect of using five different
layers for content loss. The first image portrays more of an overlaying effect
than a blending of styles. And, the last image contains almost no content,
instead blending it into the background. As such, the optimal layer must sit
some where between these. That said, "optimal" depends on the viewers taste in
images.

\begin{figure}[htp]
\centering
\caption{Samford Hall Styled as Pablo Picasso's \textit{Seated Nude} Using
Different Content Loss Layers}
\label{fig:content-layers-effect}

    \begin{minipage}{0.3\linewidth}
    \includegraphics[width=\textwidth]{img/content-layer-selection/block1_conv1}
    \captionof*{figure}{block1 conv1}
    \end{minipage}
    \begin{minipage}{0.3\linewidth}
    \includegraphics[width=\textwidth]{img/content-layer-selection/block2_conv1}
    \captionof*{figure}{block2 conv1}
    \end{minipage}
    \begin{minipage}{0.3\linewidth}
    \includegraphics[width=\textwidth]{img/content-layer-selection/block3_conv1}
    \captionof*{figure}{block3 conv1}
    \end{minipage}

\medskip

    \begin{minipage}{0.3\linewidth}
    \includegraphics[width=\textwidth]{img/content-layer-selection/block4_conv1}
    \captionof*{figure}{block4 conv1}
    \end{minipage}
    \begin{minipage}{0.3\linewidth}
    \includegraphics[width=\textwidth]{img/content-layer-selection/block5_conv1}
    \captionof*{figure}{block5 conv1}
    \end{minipage}

\end{figure}


\paragraph{Photo-realistic Style Transfer} \textit{Has there been any work to
use a real image a style and transfer live content to another content?}

\cite{gatys2016image} include a section on \textit{photo-realistic style
transfer} in the published version of \textit{A Neural Algorithm of Artistic
Style}. They find that they can transfer styles between content images with
relative success. As an example, Fig. \ref{fig:photo-realistic-style-transfer}
portrays how the algorithm can use a photo of New York at night to style a
photo of Atlanta at day, resulting in a new image of Atlanta at dusk.

\begin{figure}[htp]
\centering
\caption{Photo-realistic Style Transfer}
\label{fig:photo-realistic-style-transfer}
    \begin{minipage}{0.3\linewidth}
    \includegraphics[width=\textwidth]{img/photo-transfer/p}
    \captionof*{figure}{$\textbf{p:}$ Atlanta at Day}
    \end{minipage}
    \begin{minipage}{0.3\linewidth}
    \includegraphics[width=\textwidth]{img/photo-transfer/a}
    \captionof*{figure}{$\textbf{a:}$ New York at Night}
    \end{minipage}
    \begin{minipage}{0.3\linewidth}
    \includegraphics[width=\textwidth]{img/photo-transfer/x}
    \captionof*{figure}{$\textbf{x:}$ Atlanta at Dusk}
    \end{minipage}
\end{figure}


\paragraph{Affect of Transfer Learning} \textit{How does the choice of the
dataset affect the results?}

\paragraph{Automatic Hyperparameter Selection} \textit{Has there been any
work to find an optimal configuration of layers automatically?}

\paragraph{Marketing and Social Media} \textit{Have their been any marketing
or social media campaigns utilizing this technology?}

\paragraph{Dreamscope} \textit{Does Dreamscope (a mobile app) use this
neural method in their image filters?}

Their app store listing ambiguously states that they use cutting edge AI
techniques. However, articles confirm that at least one of their filters
utilizes Google's Deep Dream. Deep Dream, based on the "Inception" network,
utilizes the same optimization technique to style images

\paragraph{Camouflage} \textit{Are there any papers of this for hyper
localized camouflage?}

\paragraph{Style Transfer Detection} \textit{Is there any work on detecting
if this process has been applied to any image? i.e detecting the difference
between a real painting and a synthetic one?}

\paragraph{Content as a Video} \textit{When this technique is applied to
videos, how do we ensure adjacent frames are similar enough to make the video
smooth?}

\paragraph{Resource Consumption of Videos} \textit{Knowing that the current
machine (using 4x nVidia K80) is slow to process a single frame, how can we
efficiently process video at 60fps?}


%% print the bibliography using the custom NIPS bib style
\bibliographystyle{my-unsrtnat}
\bibliography{references}

% a collection of Acronyms
\begin{acronym}
\acro{CNN}{Convolutional Neural Network}
\end{acronym}

\end{document}
